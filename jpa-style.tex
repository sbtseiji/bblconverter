\bibitem{Abrams2020} Abrams, Z. (2020). \textit{Building a safe space in the pandemic}. Retrieved December 31, 2021, from\url{https://www.apa.org/topics/covid-19/pandemic-safe-space#}.
\bibitem{Adler1970} Adler, A. (1970). \textit{The education of children}. Gateway. (Original work published 1930, George Allen \textbackslash\& Unwin)
\bibitem{APA2013} American Psychiatric Association, (2013). \textit{Diagnostic and statistical manual of mental disorders}. American Psychiatric Association.
\bibitem{浅野2020} 浅野 倫子・横澤 一彦(編)  (2020). 共感覚——統合の多様性—— 横澤 一彦(監修) シリーズ統合的認知 勁草書房
\bibitem{Clement2002} Clement, E. (2002). \textit{Cognitive Flexibility: The Cornerstone of Learning}. Wiley.
\bibitem{Freud1956} Freud, S. (Ed.) (1956--1974). \textit{Standard editions of complete psychological works of Sigmund Freud} (Vols. 1-24). Hogarth Press.
\bibitem{藤永2013} 藤永 保(監修) (2013). 最新心理学事典 平凡社
\bibitem{箱田2012} 箱田 裕司(編) (2012). 認知 大山 正(監修) 心理学研究法 2 誠信書房
\bibitem{長谷川99} 長谷川 龍樹・多田 奏恵・米満 文哉・池田 鮎美・山田 祐樹・高橋 康介・近藤 洋史 (印刷中). 実証的研究の事前登録の現状と実践——OSF事前登録チュートリアル—— 心理学研究
\bibitem{長谷川2016} 長谷川 寿一・東條 正城・大島 尚・丹野 義彦・廣中 直行 (2016). はじめて出会う心理学 第3版 有斐閣
\bibitem{Helmholtz1925} von Helmholtz, H. (1925). \textit{Treatise on Physiological Optics (Vol. 3, J. P. C. Southall, Ed., \textbackslash\& Trans.). Optical Society of America. (Original work published 1910)
\bibitem{堀2009} 堀 洋道(監修)  吉田 富二雄・松井 豊・宮本 聡介(編) (2009). 新編 社会心理学 改訂版 福村出版
\bibitem{法務総合研究所2019} 法務総合研究所 (2019). 令和元年版 犯罪白書——平成の刑事政策—— 昭和情報プロセス
\bibitem{一川2016} 一川 誠 (2016). 「時間の使い方」を科学する——思考は10時から14時,記憶は16時から—— PHP研究所
\bibitem{Izard1991} Izard, C. E. (1991). \textit{The psychology of emotions}. Plenum Press.
(イザード, C. E. 荘厳 舜哉(監訳) 比較発達研究会(訳) (1996). 感情心理学 ナカニシヤ出版)
\bibitem{金政2021} 金政 祐司・古村 健太郎・浅野 良輔・荒井 崇史 (2021). 愛着不安は親密な関係内の暴力の先行要因となり得るのか?——恋愛関係と夫婦関係の縦断調査から—— 心理学研究 Advance online publication. \url{https://doi.org/10.4992/jjpsy.92.20013}
\bibitem{Katahira2020} Katahira, K., Kunisato, Y., Yamashita, Y., \& Suzuki, S. (2020). Commentary: ``A robust data-driven approach identifies four personality types across four large data sets.''. \textit{Frontiers in Big Data, 3}, 8. \url{https://doi.org/10.3389/fdata.2020.00008}
\bibitem{Katz1935} Katz, D. (1935). \textit{The world of colour} (R. B. MacLeod, \& C. W. Fox, Trans.). Kegan Paul. (Original work published 1930)
\bibitem{川上2019} 川上 直秋 (2019). 指先が変える単語の意味——スマートフォン使用と単語の感情価の関係—— 心理学研究, \textit{91}(1), 23--33. \url{https://doi.org/10.4992/jjpsy.91.18060}
\bibitem{Lamb2015} Lamb, M. E.(editor)  (2015). \textit{Socioemotional processes}. (R. M. Lerner, Series Ed.). \textit{Handbook of child psychology and developmental science}. \textit{Vol. 3}. Wiley.
\bibitem{松井2010} 松井 豊 (2010). 心理学論文の書き方——卒業論文や修士論文を書くために—— 改訂新版 河出書房新社
\bibitem{森川2010} 森川 和則 (2010). 知覚心理学は右肩下がりか,右肩上がりか——38年間のトレンド—— 心理学ワールド, 5--8.
\bibitem{Morioka2018} Morioka, M. (2018). \textit{On the constitution of self-experience in the psychotherapeutic dialogue}. In A. Konopka, H. J. M. Hermans, \& M. M. Gonçalves (Eds.), \textit{Handbook of Dialogical Self Theory and Psychotherapy: Bridging Psychotherapeutic and Cultural Traditions} (pp. 206--219). Routledge.
\bibitem{向田2009} 向田 久美子 (2009). 語りに見るライフ・スクリプトの文化心理学的研究——文化圏間比較と世代間比較を通して—— 白百合女子大学大学院博士論文
\bibitem{内藤2018} 内藤 美加 (2018). 記憶の発達と心的時間移動——自閉スペクトラム症の未解決課題再考—— 鈴木 國文・内海 健・清水 光恵(編)発達障害の精神病理I(pp. 77--96) 星和書店
\bibitem{中道2019} 中道 圭人 (2019). 幼児における他者の感情推測のための表情と身体的手がかりの利用 千葉大学教育学部研究紀要, \textit{67}, 285--292.
\bibitem{中沢1978} 中沢 潤・国本 小百合・祐宗 省三 (1978). 幼児の弁別学習——非次元性課題における過剰訓練効果—— 心理学研究, \textit{49}, 131--136.
\bibitem{日本心理学会2022} 日本心理学会 (2022). 執筆・投稿の手びき2022年版 日本心理学会 Retrieved October 25, 2022, from\url{https://psych.or.jp/manual/}.
\bibitem{野島2018} 野島 一彦・繁桝 算男(監修)(2018--2020).  公認心理師の基礎と実践(全23巻) 遠見書房
\bibitem{OSeaghdha99} O'Seaghdha, P. G. (in press). Across the great divide: Proximate units at the lexical-phonological interface. \textit{Japanese Psychological Research}.
\bibitem{Oe2016} Oe, T., Aoki, R., \& Numazaki, M. (2016). \textit{Perceived causal attributions of body temperature increase as a moderator of the effects of physical warmth on implicit associations of social warmth} [Poster presentation]. The 17th Annual Meeting of the Society for Personality and Social Psychology, San Diego, CA.
\bibitem{Osaka2007} Osaka, N., Rentschler, I., \& Biederman, I. (Eds.)  (2007). \textit{Object recognition, attention, and action}. Springer.
\bibitem{Overton2015} Overton, W. F., \& Molenaar, P. C. M. (Eds.)  (2015). \textit{Theory and Method (R. M. Lerner, Series Ed.), Handbook of Child Psychology and Developmental Science}. \textit{Vol. 1}. Wiley.
\bibitem{大谷2022} 大谷 美貴・岡田 あかり・丸山 直樹・松井 奈緒美・眞鍋 久美子・坂本 諒・山本 淳一・西村 一也・春日 麻衣・清水 真・阿部 竜太・島田 悠・谷口 幸治・上田 久美子・小林 里恵・岡本 有香・白井 太郎・鈴木 敏行・佐々木 智恵美\ldots高木 由紀子 (2022). マジカルナンバー20——20を超えると何かが起こる—— 架空心理学研究, \textit{32}(3), 80--89. \url{https://doi.org/35.3582/fmp-2pahm}
\bibitem{Rosen2015} Rosen, L. D., Cheever, N., \& Carrier, L. M. (2015). \textit{The Wiley Blackwell Handbook of Psychology, Technology and Society}. Wiley.
\bibitem{ローズ2008} Rosen, N. J. (2005). \textit{If only: How to turn regret into opportunity}. Broadway.
(ローズ, N. J. 村田 光二(監訳) (2008). 後悔を好機に変える——イフ・オンリーの心理学—— ナカニシヤ出版)
\bibitem{齊藤2019} 齊藤 慈子 (2019). 時に手を抜くイクメン,マーモセットのパパ——38年間のトレンド—— 心理学ワールド, 25--26.
\bibitem{坂本2013} 坂本 真士 (2013). 論文投稿に向けて——基礎から始める英語論文執筆—— 坂本 真士・大平 英樹(編)心理学論文道場(pp. 16--50) 世界思想社
\bibitem{サトウ2013} サトウ タツヤ (2013). ちょっとココロ学——悩み事 どうやって打開?—— 読売新聞 7月8日夕刊, 7.
\bibitem{Takahashi2017} Takahashi, N., Isaka, Y., Yamamoto, T., \& Nakamura, T. (2017). Vocabulary and Grammar Differences Between Deaf and Hearing Students. \textit{Journal of Deaf Studies and Deaf Education, 22} (1), 88--104. \url{https://doi.org/10.1093/deafed/enw055}
\bibitem{Tsukamoto2015} Tsukamoto, S. (2015). \textit{The Role of Psychological Essentialism in Intergroup Attitude Formation} (Unpublished master's thesis). Kyoto University.
\bibitem{都築2018} 都築 誉史・武田 裕司・千葉 元気 (2018). 認知資源が多肢選択意思決定における魅力効果に及ぼす影響——聴覚プローブ法を用いた実験的検討—— 日本心理学会第82回大会発表論文集, 493.
\bibitem{Wiskunde2019} Wiskunde, B., Arslan, M., Fischer, P., Nowak, L., Vanb dena Berg, O., Coetzee, L., Juárez, U., Riyaziyyat, E., Wang, C., Zhang, I., Li, P., Yang, R., Kumar, B., Xu, A., Martinez, R., McIntosh, V., Ibáñez, L. M., Mäkinen, G., Virtanen, E., \ldots Kovács, A. (2019). Indie pop rocks mathematics: Twenty One Pilots, Nicolas Bourbaki, and the empty set. \textit{Journal of Improbable Mathematics, 27} (1), 1935--1968. \url{https://doi.org/10.0000/3mp7y-537}
\bibitem{矢嶋2013} 矢嶋 美保・長谷川 晃 (2013). 家族機能が中学生の社交不安に及ぼす影響——日本の親子のデータを用いた検討—— 感情心理学研究, \textit{27}(3), 83--94. \url{https://doi.org/10.4092/jsre.27.3_83}
\bibitem{Yokoyama2020} Yokoyama, T., Kato, R., Inoue, K., \& Takeda, Y. (2020). Cuing Effects by Biologically and Behaviorally Relevant Symbolic Cues. \textit{Japanese Psychological Research} Advance online publication. \url{https://doi.org/10.1111/jpr.12318}
\bibitem{Yoshimura2021} Yoshimura, N., Morimoto, K., Murai, M., Kihara, Y., Marmolejo-Ramos, F., Kubik, V., \& Yamada, Y. (2021). \textit{Age of smile: A cross-cultural replication report of Ganel and Goodale (2018)}. PsyArXiv. \url{https://doi.org/10.31234/osf.io/dtx6}
\bibitem{坂野1994} 坂野 雄二・福井 知美・熊野 宏昭・堀江 はるみ・川原 健資・山本 晴義・野村 忍・末松 弘行 (1994). 新しい気分調査票の開発とその信頼性・妥当性の検討 心身医学, \textit{34}(8), 629--636. \url{https://doi.org/10.15064/jjpm.34.8_629}